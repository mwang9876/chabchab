\documentclass{article}
\usepackage[utf8]{inputenc}
%\usepackage[margin=1in]{geometry}

\usepackage{amsthm}
\usepackage{amsmath}
\usepackage{amsfonts}
\usepackage{amssymb}
\usepackage{graphicx}
\usepackage{bbm}

\setlength{\parindent}{0pt}
\setlength{\parskip}{3pt}
\newcommand{\lineline}{\rule{0.95\textwidth}{0.4pt}}
\newcommand{\Lineline}{\rule{\textwidth}{1pt}}
\newcommand{\boxbox}[2]{\fbox{\parbox{0.95\textwidth}{{\bf #1}#2}}}

\newcommand{\supp}{\operatorname{supp}}

\newtheorem{thm}{Theorem}[section]
\newtheorem{cor}{Corollary}[thm]
\newtheorem{lemma}{Lemma}
\newtheorem{defn}{Definition}
\newtheorem{fact}{Fact}

%%dark mode
\usepackage{xcolor}
\usepackage{pagecolor}
%\pagecolor{black}
%\color{white}


\begin{document}

\section{Chabauty space of $\text{PSL}_2(\mathbb{R})$}

Let $G = \text{PSL}_2(\mathbb{R})$. The group $G$ acts on $\mathbb{H}^2$ by fractional linear transformation on upper half plane. An elemenatary subgroup of $G$ is a subgroup of $G$ that has a finite orbit in $\mathbb{H}^2 \cup \partial\mathbb{H}^2$.

Let the set $A$ be a finite orbit of an elementary subgroup $H \leq G$. If $H$ contains an element that is parabolic or hyperbolic, then $A$ does not contain any element in $\mathbb{H}^2$. So in the case where $A$ is contained in $\mathbb{H}^2$, we know that $H$ can only have elliptic elements. The following fact is from Katok, where it is proved algebraically. 

\begin{fact}
	If $H$ contains only elliptic elements, then it has a fixed point in $\mathbb{H}^2$.
\end{fact}





\end{document}
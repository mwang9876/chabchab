\documentclass{article}
\usepackage[utf8]{inputenc}
%\usepackage[margin=1in]{geometry}

\usepackage{amsthm}
\usepackage{amsmath}
\usepackage{amsfonts}
\usepackage{amssymb}
\usepackage{graphicx}
\usepackage{bbm}

\setlength{\parindent}{0pt}
\setlength{\parskip}{3pt}
\newcommand{\lineline}{\rule{0.95\textwidth}{0.4pt}}
\newcommand{\Lineline}{\rule{\textwidth}{1pt}}
\newcommand{\boxbox}[2]{\fbox{\parbox{0.95\textwidth}{{\bf #1}#2}}}

\newcommand{\supp}{\operatorname{supp}}

\newtheorem{thm}{Theorem}[section]
\newtheorem{cor}{Corollary}[thm]
\newtheorem{lemma}{Lemma}
\newtheorem{defn}{Definition}
\newtheorem{fact}{Fact}

\bibliographystyle{alpha}
\bibliography{ref}

%%dark mode
\usepackage{xcolor}
\usepackage{pagecolor}
%\pagecolor{black}
%\color{white}


\begin{document}

\section{Chabauty space of $\text{PSL}_2(\mathbb{R})$}

Let $G = \text{PSL}_2(\mathbb{R})$. The group $G$ acts on $\mathbb{H}^2$ by fractional linear transformation on upper half plane. An elemenatary subgroup of $G$ is a subgroup of $G$ that has a finite orbit in $\mathbb{H}^2 \cup \partial\mathbb{H}^2$.

Let the set $A$ be a finite orbit of an elementary subgroup $H \leq G$. 
First we can think about where the set $A$ is and what element $H$ can have.
If $H$ contains an element that is parabolic or hyperbolic, then $A$ cannot contain any element in $\mathbb{H}^2$. Otherwise the parabolic or hyperbolic isometry will create an infinite orbit. So in the case where $A$ is contained in $\mathbb{H}^2$, we know that $H$ can only have elliptic elements. The following fact is from [Katok], where it is proved algebraically. 

\begin{fact} 
	If $H$ contains only elliptic elements, then it has a fixed point in $\mathbb{H}^2$.
\end{fact}

The fact above tells us that if elementary subgroup $H$ only has elliptical elements, then every element fixes the same point. In other words, all the elements are some rotation around that fixed point. 

We want to argue that if $h \in H$ is parabolic or hyperbolic, then $h$ also creates an infinite orbit of any $\xi \in \partial \mathbb{H}^2$ that is not a fixed point of $h$. For example, in upper half plane model, if $h$ is the map $z \mapsto \lambda z$, it is a hyperbolic isometry leaving the imaginary axis invariant. Any other hyperbolic isometry is a conjugation of $h$. The map $h$ has two fixed points $0$ and $\infty$, for any nonzero $\xi$ on the real line, the orbit is $\{\lambda^n \xi:\ n\in \mathbb{Z}\}$. Similarly if $h$ is the map $z \mapsto z+1$, it is a parabolic isometry fixing $\infty$, and any $\xi \in \mathbb{R}$ has infinite orbit $\{\xi + n:\ n\in \mathbb{Z}\}$ under $\langle h \rangle$. Any other parabolic map is a conjugation to $h$. 

Knowing how different types of isometry determines what elements $A$ can have, we can begin to classify all the closed elementary subgroup $H \leq G$. 
The following result is from [BLL21].

\begin{thm}
	The closed, elementary subgroups of $G$ are as follows.
	\begin{enumerate}
		\item The trivial group.
		\item The group $K(p)$ of all elliptic isometries fixing some $p \in \mathbb{H}^2$, and the finite cyclic subgroup $k(p,2\pi/n)$ generated by a $2\pi/n$-rotation around $p \in \mathbb{H}^2$.
		\item The group $N(\xi)$ of all parabolic isometries fixing some $\xi \in \partial \mathbb{H}^2$, and its infinite cyclic subgroup. 
		\item The group $A(\alpha)$ of all hyperbolic type isometries that translate along geodesic $\alpha \subseteq \mathbb{H}^2$, and its infinite cyclic subgroup $a(\alpha,t)$ consisting of translations by multiple of $t \in \mathbb{R}$.
		\item The group $A'(\alpha) \cong A(\alpha) \rtimes \mathbb{Z}/2\mathbb{Z}$ and its infinite dihedral subgroup $a'(\alpha,t,p)\cong a(\alpha,t)\rtimes \mathbb{Z}/2\mathbb{Z}$. Here the subgroup $\mathbb{Z}/2\mathbb{Z}$ is generated by a $\pi$-rotation around $p \in \alpha$. In the latter case, $p$ is well-defined up to translations along $\alpha$ by multiple of $t$. 
		\item The group $B(\xi)\cong N(\xi)\rtimes A(\alpha)$ of isometries fixing $\xi \in \partial \mathbb{H}^2$, and its subgroup $b(\xi,t) \cong N(\xi) \rtimes a(\alpha,t)$. Here $\alpha$ is a geodesic with $\xi$ as one end. 
	\end{enumerate}
\end{thm}

\begin{proof}
	We will give a proof that looks different from the one in [BLL21]. We consider what kind of isometries plays well with each other and can coexists in the same subgroup $H$ while still having finite orbit. Figure \ref{fig:1.1} summarizes the result.
	\begin{figure}
		\centering
		\includegraphics[width = 0.7\textwidth]{IMG_0133.jpg}
		\caption{Compatibility chart between isometries}
		\label{fig:1.1}
	\end{figure}
\end{proof}
	


\end{document}
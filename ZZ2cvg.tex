\documentclass{article}
\usepackage{amsmath, amssymb}
\usepackage{amsfonts}
\usepackage{graphicx}
\usepackage{lpic}
\usepackage{amsthm}
\usepackage{float}
\usepackage{amsthm}
\usepackage{mathtools}
\usepackage[top=1in, bottom=1in, left=1.5in, right=1.5in]{geometry}
\usepackage{etoolbox}
\patchcmd{\quote}{\rightmargin}{\leftmargin 2em \rightmargin}{}{}
\allowdisplaybreaks

\usepackage{hyperref}
\hypersetup{pdfauthor={},pdftitle={determining the subgraphs of the curve graph},colorlinks=true,linkcolor=black,citecolor=black}


\newtheorem{theorem}{Theorem}[section]
\newtheorem{lemma}[theorem]{Lemma}
\newtheorem{fact}[theorem]{Fact}
\newtheorem{proposition}[theorem]{Proposition}
\newtheorem{corollary}[theorem]{Corollary}
\newtheorem{conjecture}[theorem]{Conjecture}
\newtheorem{notation}[theorem]{Notation}
\newtheorem*{theorem*}{Theorem}
\theoremstyle{remark}
\newtheorem{remark}[theorem]{Remark}
\newtheorem{definition}[theorem]{Definition}
\newtheorem{example}[theorem]{Example}
\newtheorem{question}{Question}






%this has equations numbered within sections 1.1,1.2, ... 2.1,...
\numberwithin{equation}{section}


%-------------------------------------------
%       Begin Local Macros
%-------------------------------------------

\newcommand{\bC}{\mathbb{C}}
\newcommand{\bH}{\mathbb{H}}
\newcommand{\bN}{\mathbb{N}}
\newcommand{\bQ}{\mathbb{Q}}
\newcommand{\bR}{\mathbb{R}}
\newcommand{\bT}{\mathbb{T}}
\newcommand{\bZ}{\mathbb{Z}}
\newcommand{\cC}{\mathcal{C}}
\newcommand{\cI}{\mathcal{I}}
\newcommand{\cP}{\mathcal{P}}
\newcommand{\Isom}{\mathrm{Isom}}
\newcommand{\PSL}{\mathrm{PSL}}
\newcommand{\Mod}{\mathrm{Mod}}        

%-------------------------------------------
%       End Local Macros
%-------------------------------------------

\linespread{1.1}

\title{Why can't I make the title look nice}
\author{Mujie Wang}
\date{\today}

\begin{document}
\maketitle

\begin{abstract}
	Future me will put some abstract here
\end{abstract}

\section{Introduction}



\section{Chab Chab}

Let $G$ be a Lie group, and $A_n$ and $A$ be its closed subgroups. We want to find a notion of convergence for closed subgroups. One such notion is Chabauty topology. 

\begin{definition} \label{def:chabconv}
A sequence of subgroups $(A_n)$ converge to $A$ if 
\begin{enumerate}
	\item Any convergent sequence $(x_n)$ where $x_n \in A_n$ will converge to some element in $x \in A$.
	\item If $x \in A$, then there exists some subsequence $(x_{n_k})$ where $x_{n_k} \in A_{n_k}$ such that it converge to $x$. In other words, $x$ is an accumulation point. 
\end{enumerate}
\end{definition}

For this section, we let $G = \PSL_2(\bR)$. The group $G$ is the group of isometries acting on the upper half plane model of $\bH^2$ by fractional linear transformation. Using Chabauty topology, we are able to describe the shape of the collection of all closed subgroups of $G$, denoted by $S(G)$. To do so, We want to break $S(G)$ into pieces that we know the shape of, and then try to glue it back together.

A sequence of cyclic subgroup generated by a hyperbolic isometry can converge to a cyclic subgroup generated by a parabolic isometry. 
\begin{example}
Let $\xi_n = 1 + 1/n$,
Consider 
$$H_n = \dfrac12
\begin{bmatrix} 
\xi_n + \xi_n^{-1} & n(-\xi_n + \xi_n^{-1}) \\
\dfrac1n(-\xi_n + \xi_n^{-1}) & \xi_n + \xi_n^{-1}
\end{bmatrix}, \ \ 
P = 
\begin{bmatrix}
1 & -1 \\ 0 & 1
\end{bmatrix}
$$
The sequence of subgroups $(\langle H_n \rangle)$ converges to the subgroup $\langle P \rangle$.
\end{example}

Here, $H_n$ is a hyperbolic isometry that fixes 

\section{The plot thickens}


Jorgenson first discovered the following interesting phenomenon.

\begin{fact} \label{fact:Jorgenson}
	There exists a sequence of infinite cyclic groups $\langle H_n \rangle$, where $H_n$ is an isometry of hyperbolic type, such that the sequence converges to a subgroup isomorphic to $\mathbb{Z}^2$, whose generators are both parabolic isometries.
\end{fact}

In this section, we will construct one such sequence explicitly. We will use the upper half-plane model $\bC \times \bR^+$ for $\bH^3$. The isometry group $\cI^+(\bH^3) \cong \PSL_2(\bC)$ acts on the boundary $\overline\bC$ by fractional linear transformation. 

\section{Quotient things up}

\section{Convergence of the quotient}

\section{Closet of necessary backgrounds}

\bigskip

\bibliographystyle{alpha}
\newcommand{\etalchar}[1]{$^{#1}$}
\begin{thebibliography}{OeSHP14}

\bibitem[PH]{PH}
Author names.
\newblock Article Titles.
\newblock {\em Journal Title},
  ??? some mysterious numbers.


\end{thebibliography}








\end{document}
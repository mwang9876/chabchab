\documentclass{article}
\usepackage{amsmath, amssymb}
\usepackage{amsfonts}
\usepackage{graphicx}
\usepackage{lpic}
\usepackage{amsthm}
\usepackage{float}
\usepackage{amsthm}
\usepackage{mathtools}
\usepackage[top=1in, bottom=1in, left=1.5in, right=1.5in]{geometry}
\usepackage{etoolbox}
\patchcmd{\quote}{\rightmargin}{\leftmargin 2em \rightmargin}{}{}
\allowdisplaybreaks

\usepackage{hyperref}
\hypersetup{pdfauthor={},pdftitle={determining the subgraphs of the curve graph},colorlinks=true,linkcolor=black,citecolor=black}


\newtheorem{theorem}{Theorem}[section]
\newtheorem{lemma}[theorem]{Lemma}
\newtheorem{fact}[theorem]{Fact}
\newtheorem{proposition}[theorem]{Proposition}
\newtheorem{corollary}[theorem]{Corollary}
\newtheorem{conjecture}[theorem]{Conjecture}
\newtheorem{notation}[theorem]{Notation}
\newtheorem*{theorem*}{Theorem}
\theoremstyle{remark}
\newtheorem{remark}[theorem]{Remark}
\newtheorem{definition}[theorem]{Definition}
\newtheorem{example}[theorem]{Example}
\newtheorem{question}{Question}






%this has equations numbered within sections 1.1,1.2, ... 2.1,...
\numberwithin{equation}{section}


%-------------------------------------------
%       Begin Local Macros
%-------------------------------------------

\newcommand{\bC}{\mathbb{C}}
\newcommand{\bH}{\mathbb{H}}
\newcommand{\bN}{\mathbb{N}}
\newcommand{\bQ}{\mathbb{Q}}
\newcommand{\bR}{\mathbb{R}}
\newcommand{\bT}{\mathbb{T}}
\newcommand{\bZ}{\mathbb{Z}}
\newcommand{\cC}{\mathcal{C}}
\newcommand{\cI}{\mathcal{I}}
\newcommand{\cP}{\mathcal{P}}
\newcommand{\PSL}{\mathrm{PSL}}
\newcommand{\Mod}{\mathrm{Mod}}        

%-------------------------------------------
%       End Local Macros
%-------------------------------------------

\linespread{1.1}

\title{Why can't I make the title look nice}
\author{Mujie Wang}
\date{\today}

\begin{document}
\maketitle

\begin{abstract}
	Future me will put some abstract here
\end{abstract}

\section{Introduction}

\section{Annex of necessary backgrounds}

\section{Two examples in $\PSL_2(\bR)$ (that will help us later)}

\section{The plot thickens}


Jorgenson first discovered the following interesting phenomenon.

\begin{fact} \label{fact:2.1}
	There exists a sequence of infinite cyclic groups $\langle H_n \rangle$, where $H_n$ is an isometry of hyperbolic type, such that the sequence converges to a subgroup isomorphic to $\mathbb{Z}^2$, whose generators are both parabolic isometries.
\end{fact}

In this section, we will construct one such sequence explicitly. We will use the upper half-plane model $\bC \times \bR^+$ for $\bH^3$. The isometry group $\cI^+(\bH^3) \cong \PSL_2(\bC)$ acts on the boundary $\overline\bC$ by fractional linear transformation. 



\bigskip

\bibliographystyle{alpha}
\newcommand{\etalchar}[1]{$^{#1}$}
\begin{thebibliography}{OeSHP14}

\bibitem[PH]{PH}
Author names.
\newblock Article Titles.
\newblock {\em Journal Title},
  ??? some mysterious numbers.


\end{thebibliography}








\end{document}